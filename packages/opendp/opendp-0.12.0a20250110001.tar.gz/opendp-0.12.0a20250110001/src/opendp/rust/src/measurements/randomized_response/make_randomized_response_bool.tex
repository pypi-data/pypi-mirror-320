\documentclass{article}
% common styling and macros shared by all proof files

\usepackage[top=1in, right=1in, left=1in, bottom=1.5in]{geometry}

\usepackage{amsmath,amsthm,amsfonts,amssymb,amscd}
\usepackage{listings}
\usepackage{hyperref}
\usepackage{xcolor}
\usepackage{xr}

\usepackage{enumerate} 
\usepackage{physics}
\usepackage{fancyhdr}
\usepackage{hyperref}
\usepackage{graphicx}
\usepackage{tcolorbox}
\usepackage{catchfile}
\usepackage{pdftexcmds}
\usepackage[T1]{fontenc}

% hyperref
\hypersetup{
  colorlinks=true,
  linkcolor=blue,
  linkbordercolor={0 0 1}
}

% \contrib macro to indicate inclusion in "contrib".
\usepackage{tcolorbox}
\newtcolorbox{warn_box}{colback=red!5!white,colframe=red!75!black}
\newcommand{\contrib}{{\begin{warn_box}This proof resides in \textbf{``contrib''} because it has not completed the vetting process.\end{warn_box}}} 
\newcommand{\floatingPoint}{{\begin{warn_box}This implementation is susceptible to floating-point vulnerabilities.\end{warn_box}}} 

% asOfCommit macro to version a code dependency. Arguments:
%    #1: relative path to file you are dependent on
%    #2: commit hash it was last edited. If outdated, this should be the second hash in the footnoote. Otherwise,
%            git log -n 1 --pretty=format:%h -- path/to/file.rs
\makeatletter
\ifnum\pdf@shellescape=1
   % "private" command that builds a link to a blob
  \newcommand{\linkOpendpBlob}[3]{%
    \href{https://github.com/opendp/opendp/blob/#1/#2#3}{\path{#3} at commit #1}}

  % latex macro expansion has a separate phase for \input evaluation
  %     immediately evaluate a command to write a temp file to ./out containing the current directory
  \immediate\write18{[ ! -f out/cwd.txt ] && (mkdir -p out && git rev-parse --show-prefix | sed "s|_|\@backslashchar\@backslashchar\@backslashchar_|g" > out/cwd.txt)}
  %     ...and then retrieve the current working directory by loading the temp file
  \CatchFileDef\GitWorkingDir{out/cwd.txt}{\endlinechar=-1}

  % command for building the (up to date) or (outdated) status
  \newcommand{\fileStatus}[2]{%
  \setbox0=\hbox{\input|"git --no-pager log -n1 --pretty='\@percentchar H' #1 | grep -E '^#2.*'"\unskip}\ifdim\wd0=0pt
        (outdated\footnote{See new changes with \texttt{git diff #2..\input|"git --no-pager log -n1 --pretty='\@percentchar h' #1" \GitWorkingDir\path{#1}}})\else
        (up to date)\fi
  }

  \newcommand{\asOfCommit}[2]{%
      % permalink the target
      \linkOpendpBlob{#2}{\GitWorkingDir}{#1}
      % conditionally add (outdated) or (up to date) depending on matching commit hash
      \fileStatus{#1}{#2}%
  }
\else
  % simplified command if shell-escape not enabled
  \newcommand{\asOfCommit}[2]{#1 at commit #2 (unknown status\footnote{Shell-escape is not enabled. Enable \texttt{--shell-escape} to check if this proof is up-to-date with the code.})}
\fi
\makeatother

% \vettingPR macro to link a PR. Arguments:
%    #1: PR number
\newcommand{\vettingPR}[1]{\href{https://github.com/opendp/opendp/pull/#1}{Pull Request \##1}}

% for links to rustdoc items in OpenDP. Arguments:
%    #1: path to item, and designation as trait, struct, fn, etc.
%    #2: item name
\makeatletter
\ifnum\pdf@shellescape=1
  % latex macro expansion has a separate phase for \input evaluation
  %     immediately evaluate a command to write a temp file to ./out containing the base path
  \immediate\write18{[ ! -f out/rustdoc.txt ] && mkdir -p out && ([ -z `kpsewhich --var-value OPENDP_RUSTDOC_PORT` ] && echo "https://docs.rs/opendp/`head -n 1 \@backslashchar`git rev-parse --show-toplevel\@backslashchar`/VERSION | sed 's|.*-dev.*|latest|g'`" || echo "http://localhost:`kpsewhich --var-value OPENDP_RUSTDOC_PORT`") > out/rustdoc.txt}
  %     ...and then retrieve the base path by loading the temp file
  \CatchFileDef\OpenDPRustdocBase{out/rustdoc.txt}{\endlinechar=-1}
\else
  % if shell commands are not enabled, just claim latest
  \newcommand{\OpenDPRustdocBase}{https://docs.rs/opendp/latest}
\fi
\makeatother
\newcommand{\rustdoc}[2]{\href{\OpenDPRustdocBase/opendp/#1.#2.html}{\texttt{#2}}}

% for links to external dependencies. Arguments:
%    #1: crate name
%    #2: path to item, and designation as trait, struct, fn, etc.
%    #3: item name
\newcommand{\docsrs}[3]{\href{https://docs.rs/#1/latest/#1/#2.#3.html}{\texttt{#3}}}

% minted (pseudocode)
\definecolor{codegreen}{rgb}{0,0.6,0}
\definecolor{codegray}{rgb}{0.5,0.5,0.5}
\definecolor{codepurple}{rgb}{0.58,0,0.82}
\definecolor{backcolour}{rgb}{0.95,0.95,0.92}

\lstdefinestyle{mystyle}{
    backgroundcolor=\color{backcolour},   
    commentstyle=\color{codegreen},
    keywordstyle=\color{magenta},
    numberstyle=\tiny\color{codegray},
    stringstyle=\color{codepurple},
    basicstyle=\ttfamily\footnotesize,
    breakatwhitespace=false,         
    breaklines=true,                 
    captionpos=b,                    
    keepspaces=true,                 
    numbers=left,                    
    numbersep=5pt,                  
    showspaces=false,                
    showstringspaces=false,
    showtabs=false,                  
    tabsize=2
}

\lstset{style=mystyle}

% common commands
\theoremstyle{definition}
\newtheorem{theorem}{Theorem}[section]
\newtheorem{lemma}[theorem]{Lemma}
\newtheorem{definition}[theorem]{Definition}
\newtheorem{warning}{Warning}
\newtheorem{corollary}{Corollary}
\newtheorem{proposition}{Proposition}
\newtheorem{remark}{Remark}
\newtheorem{observation}{Observation}
\newtheorem{note}{Note}

\newcommand{\vicki}[1]{{ {\color{olive}{(vicki)~#1}}}}
\newcommand{\hanwen}[1]{{ {\color{purple}{(hanwen)~#1}}}}
\newcommand{\zach}[1]{{ {\color{red}{(zach)~#1}}}}

\newcommand{\MultiSet}{\mathrm{MultiSet}}
\newcommand{\len}{\mathrm{len}}
\newcommand{\din}{\texttt{d\_in}}
\newcommand{\dout}{\texttt{d\_out}}
\newcommand{\T}{\texttt{T} }
\newcommand{\F}{\texttt{F} }
\newcommand{\Map}{\texttt{Map}}
\newcommand{\X}{\mathcal{X}}
\newcommand{\Y}{\mathcal{Y}}
\newcommand{\True}{\texttt{True}}
\newcommand{\False}{\texttt{False}}
\newcommand{\clamp}{\texttt{clamp}}
\newcommand{\function}{\texttt{function}}
\newcommand{\float}{\texttt{float }}
\newcommand{\questionc}[1]{\textcolor{red}{\textbf{Question:} #1}}


\newcommand{\validTransformation}[2]{%
  \begin{theorem}
  For every setting of the input parameters #1 to #2 such that the given preconditions
  hold, #2 raises an exception (at compile time or run time) or returns a valid transformation. A valid transformation has the following properties:
  \begin{enumerate}
      \item \textup{(Appropriate output domain).} 
      For every element $x$ in \texttt{input\_domain}, $\function(x)$ is in \texttt{output\_domain} or raises a data-independent runtime exception.
      
      \item \textup{(Stability guarantee).} 
      For every pair of elements $x, x'$ in \texttt{input\_domain} and for every pair $(\din, \dout)$, 
      where \din\ has the associated type for \texttt{input\_metric} and \dout\ has the associated type for \\ \texttt{output\_metric}, 
      if $x, x'$ are \din-close under \texttt{input\_metric}, $\texttt{stability\_map}(\din)$ does not raise an exception,
      and $\texttt{stability\_map}(\din) \leq \dout$, 
      then $\function(x), \function(x')$ are $\dout$-close under \texttt{output\_metric}.
  \end{enumerate}
  \end{theorem}
}


\newcommand{\validMeasurement}[2]{%
  \begin{theorem}
  For every setting of the input parameters #1 to #2 such that the given preconditions
  hold, #2 raises an exception (at compile time or run time) or returns a valid measurement. A valid measurement has the following property:
  \begin{enumerate}
      \item \textup{(Privacy guarantee).}
      For every pair of elements $x, x'$ in \texttt{input\_domain} and for every pair $(\din, \dout)$,
      where \din\ has the associated type for \texttt{input\_metric} and \dout\ has the associated type for \\ \texttt{output\_measure},
      if $x, x'$ are \din-close under \texttt{input\_metric}, $\texttt{privacy\_map}(\din)$ does not raise an exception,
      and $\texttt{privacy\_map}(\din) \leq \dout$,
      then $\function(x), \function(x')$ are $\dout$-close under \texttt{output\_measure}.
  \end{enumerate}
  \end{theorem}
}




\title{\texttt{fn make\_randomized\_response\_bool}}
\author{Vicki Xu, Hanwen Zhang, Zachary Ratliff}
\begin{document}

\maketitle

\contrib

Proves soundness of \rustdoc{measurements/fn}{make\_randomized\_response\_bool} in \asOfCommit{mod.rs}{f5bb719}.

\texttt{make\_randomized\_response\_bool} accepts a parameter \texttt{prob} of type \texttt{Q} and a parameter \texttt{constant\_time} of type \texttt{bool}.
The function on the resulting measurement takes in a boolean data point \texttt{arg} and returns the truthful value \texttt{arg} with probability \texttt{prob},
or the complement $\texttt{!arg}$ with probability $1 - \texttt{prob}$.
The measurement function makes mitigations against timing channels if \texttt{constant\_time} is set. 

\begin{tcolorbox}
    \begin{warning}[Code is not constant-time]
        \texttt{make\_randomized\_response\_bool} takes in a boolean \texttt{constant\_time} parameter that protects against timing attacks on the Bernoulli sampling procedure. 
        However, the current implementation does not guard against other types of timing side-channels that can break differential privacy, e.g., non-constant time code execution due to branching.
    \end{warning}
\end{tcolorbox}

\subsection*{PR History}
\begin{itemize}
    \item \vettingPR{490}
\end{itemize}

\section{Hoare Triple}

\subsection*{Preconditions}
\begin{itemize}
    \item Variable \texttt{prob} must be of type \texttt{QO}
    \item Variable \texttt{constant\_time} must be of type \texttt{bool}
    \item Type \texttt{bool} must have trait \rustdoc{traits/samplers/bernoulli/trait}{SampleBernoulli}\texttt{<QO>}
    \item Type \texttt{QO} must have trait \rustdoc{traits/trait}{Float}
\end{itemize}

\subsection*{Pseudocode}
\lstinputlisting[language=Python,firstline=2,escapechar=|]{./pseudocode/make_randomized_response_bool.py}

\subsection*{Postcondition}

\validMeasurement{\texttt{(prob, constant\_time, QO)}}{\\ \texttt{make\_randomized\_response\_bool}}

\section{Proof}

\begin{proof} 
\textbf{(Privacy guarantee.)} 

\begin{tcolorbox}
\begin{note}
    The following proof makes use of the following lemma that asserts the correctness of a Bernoulli sampler function.
    \begin{lemma}
        \texttt{sample\_bernoulli\_float} satisfies its postcondition.
    \end{lemma}
\end{note}
\end{tcolorbox}

\texttt{sample\_bernoulli} can only fail due to lack of system entropy. 
This is usually related to the computer's physical environment and not the dataset. 
The rest of this proof is conditioned on the assumption that \texttt{sample\_bernoulli} does not raise an exception. 

Let $x$ and $x'$ be datasets in the input domain (either $\top$ or $\bot$) that are \texttt{d\_in}-close with respect to \texttt{input\_metric}.
Here, the metric is \texttt{DiscreteMetric} which enforces that $\din \geq 1$ if $x \ne x'$ and $\din = 0$ if $x = x'$. 
If $x = x'$, then the output distributions on $x$ and $x'$ are identical, and therefore the max-divergence is 0.
Consider $x \ne x'$ and assume without loss of generality that $x = \top$ and $x' = \bot$. 
For shorthand, we let $p$ represent \texttt{prob}, the probability that \texttt{sample\_bernoulli\_float} returns $\top$. 
Observe that $p = [0.5, 1.0)$ otherwise \texttt{make\_randomized\_response\_bool} raises an error. 

We now consider the max-divergence $D_{\infty}(Y||Y')$ over the random variables $Y = \function(x)$ and $Y' = \function(x')$.

\begin{align*}
    \max_{x \sim x'} D_{\infty}(Y||Y')
    =& \max_{x \sim x'} \max_{S \subseteq Supp(Y)}\ln (\frac{\Pr[Y \in S]}{\Pr[Y' \in S]}) \\
    \le& \max_{x \sim x'} \max_{y \in Supp(Y)}\ln (\frac{\Pr[Y = y]}{\Pr[Y' = y]}) &&\text{Lemma 3.3 } \cite{Kasiviswanathan_2014} \\
    =& \max_{x \sim x'} \max\left(\ln (\frac{\Pr[Y = \top]}{\Pr[Y' = \top]}), \ln(\frac{\Pr[Y = \bot]}{\Pr[Y' = \bot]})\right) \\
    =& \max\left(\ln (\frac{p}{1 - p}), \ln(\frac{1 - p}{p})\right) \\
    =& \ln (\frac{p}{1 - p})
\end{align*}

We let $c = \texttt{privacy\_map}(\din) = \texttt{QO.inf\_ln(prob.inf\_div(1.neg\_inf\_sub(prob)))}$.
The computation of \texttt{c} rounds upward in the presence of floating point rounding errors. 
This is because \texttt{1.neg\_inf\_sub(prob)} appears in the denominator, and to ensure that the bound holds even in the presence of rounding errors, the conservative choice is to round down (so the quantity as a whole is bounded above). 
Similarly, \texttt{inf\_div} and \texttt{inf\_ln} round up. 

When $\din > 0$ and no exception is raised in computing $\texttt{c} = \texttt{privacy\_map}(\din)$, then $\ln\left(\frac{p}{1 - p}\right) \leq \texttt{c}$. 

Therefore we've shown that for every pair of elements $x, x' \in \{\bot, \top\}$ and every $d_{DM}(x, x') \le \din$ with $\din \ge 0$, 
if $x, x'$ are $\din$-close then $\function(x),\function(x') \in \{\bot, \top\}$ are $\texttt{privacy\_map}(\din)$-close under $\texttt{output\_measure}$ (the Max-Divergence).
\end{proof}


\bibliographystyle{plain}
\bibliography{randomized_response.bib}

\end{document}
